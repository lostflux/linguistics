\section{CFGs for Generating English Haiku}

\subsection{Generating Haiku}

\begin{enumarabic}
  \item My first attempt at generating Haikus was to simply generate sequences of words
    such that the words in the first, second, and third line have $5$, $7$, and $5$ syllables respectively.
    I came up with this basic CFG:

    \begin{center}
      \begin{small}
        \begin{verbatim}
          S -> SOMETHING1 BREAK SOMETHING2 BREAK SOMETHING3
            
          SOMETHING1 -> W1 W4 | W2 W3 | W3 W2 | W4 W1
          SOMETHING2 -> W3 W2 W3 | W4 W1 W2
          SOMETHING3 -> W4 W1 | W3 W2 | W1 W4 | W2 W3
          
          BREAK -> "|"
          
          W1 -> { " | ".join([ f"`{word}'" for word in W1 ]) }
          W2 -> { " | ".join([ f"`{word}'" for word in W2 ]) }
          W3 -> { " | ".join([ f"`{word}'" for word in W3 ]) }
          W4 -> { " | ".join([ f"`{word}'" for word in W4 ]) }
        \end{verbatim} 
      \end{small}
    \end{center}

    \begin{center}
      \begin{small}
        \begin{verbatim}
          # example generated
          0          : key important | dangerous sister dangerous | important key
          10000000   : be important | impossible do mother | sister mysterious
          83000000   : mother remember | appreciate big answer | incredible key
          118000000  : teacher happy | consider answer teacher | impossible be
          180000000  : impossible do | impossible big whisper | silly dangerous
        \end{verbatim} 
      \end{small}
    \end{center}
    The key shortfall of this approach is that \emph{most} generated
    sentences do not making syntactic (leave alone semantic) sense.
    And that is to be expected --- proper English sentences have similar
    structures with nouns, verbs, adjectives, and adverbs in specific
    positions, and this CFG does not enforce that.
  \item A first improvement was making the \verb|CFG| generate sentences
    whose syllable-count add up to the right counts, but also try to retain
    some semblance of English syntax by enforcing that specific types of words
    be positioned in specific places in the sentence.
    The other thing I realized is that Haikus are more likely to contain
    certain words (describing animals, nature, seasons, etc.) than they are to contain
    words such as ``important'', ``appreciate'', ``remember'', etc.
    So I also changed the terminals to be more Haiku-like.
    This is the CFG I came up with:

    \begin{center}
      \begin{small}
        \begin{verbatim}
          S -> NP-5 BREAK S-7 BREAK S-5
          
          NP-5 -> DET-1 ADJ-1 ADJ-2 NOUN-1
          S-7 -> VP-3 NPOBJ-4
          S-5 -> S-1 NPOBJ-4
          
          VP-3 -> NP-2 VERB-1
          NPOBJ-4 -> PREP-1 NP-3
          
          S-1 -> NOUN-1
          
          NP-2 -> DET-1 NOUN-1
          NP-3 -> DET-1 ADJ-1 NOUN-1 | DET-1 NOUN-2
          
          BREAK -> "|"
          
          DET-1 -> { " | ".join([ f"'{word}'" for word in DET1 ]) }
          ADJ-1 -> { " | ".join([ f"'{word}'" for word in ADJ1 ]) }
          ADJ-2 -> { " | ".join([ f"'{word}'" for word in ADJ2 ]) }
          PREP-1 -> { " | ".join([ f"'{word}'" for word in PREP1 ]) }
          CONJ-1 -> { " | ".join([ f"'{word}'" for word in CONJ1 ]) }
          ADV-1 -> { " | ".join([ f"'{word}'" for word in ADV1 ]) }
          NOUN-1 -> { " | ".join([ f"'{word}'" for word in NOUN1 ]) }
          PNOUN-1 -> { " | ".join([ f"'{word}'" for word in PNOUN1 ]) }
          VERB-1 -> { " | ".join([ f"'{word}'" for word in VERB1 ]) }  
        \end{verbatim}
      \end{small}
    \end{center}

    \begin{center}
      \begin{small}
        \begin{verbatim}
# example generated
0          : an old silent frog | an frog splash with an old frog | frog with an old frog
7700000    : an old silent pond | a frog jumps in the old pond | frog of the old frog
10100000   : an old silent pond | the dew splash in the old pond | world from the old frog
12900000   : an old silent pond | a world splash of an old dew | dew from an old frog
24600000   : an old silent world | the frog jumps in the old pond | frog in an old pond
        \end{verbatim} 
      \end{small}
    \end{center}
    This version shows some notable improvements: some sentences
    actually make sense! But as expected, the majority of sentences do not.
    A clear shortfall is also that most Haikus will have some relationship
    between the lines (similar words, idea continuation, etc.).
    This CFG does not intentianally enforce that.
    Consequently, even when each individual line makes sense,
    most times the haiku as a whole does not.
    Some successful idea connections that I noticed are:
    \begin{enumarabic}
      \item \verb#an old silent pond | a frog jumps into the old pond | frog of the old pond#
      \item \verb#an old silent world | the frog jumps into the old pond | frog in an old pond#
    \end{enumarabic}

    I attempted adding more terminal words to the CFG, but that did not improve the quality of the generated haikus
    --- in fact, it made it worse. Because the range of words was too big, the first sentence
    would never change (since the generator seemed to combinatorially try all possible variations of sentences,
    swapping words from the end of the sentence).

    Because of this, I realized that it is impossible to improve the performance
    without the unrealistic task of manually defining a lot of rules enforcing
    that \emph{"if this noun is selected in sentence 1, then this verb must be selected in sentence 2
    and this adjective in sentence 3"}, etc.

    Perhaps having structures such as ``capture groups'' from regular expressions
    would have made it possible to explicitly reuse words from the first sentence
    in the second and third sentence and skip the millions of combinations that
    do not have the desired property.

  \item the favorite $3$ haikus generated that I thought were among the most successful
    in making sense were:
    \begin{enumroman}
      \item \emph{an old silent world | the frog jumps in the old pond | frog in the old pond}
      \item \emph{the old peaceful pond | the dew splashes in the pond | dew in the old pond}
      \item \emph{a new peaceful world | the frog splashes in the pond | an old peaceful world}
    \end{enumroman}
    Note: the haikus appear similar because I manually reduced the set of words so I can
    get more interesting variations (when the words were too many, as mentioned above,
    the generator would barely change the first sentence and parts of the second
    because there were millions of variations for the last sentence alone).
    You can reproduce this behavior by un-commenting the commented-out words in the
    word-sets used in the CFG (see the code notebook).
\end{enumarabic}

\newpage
\subsection{Writing an Actual Haiku: incorporating \crim{kigo} and \crim{kireji}}

There are several possible ways to enforce the inclusion
of a \crim{kigo} and \crim{kireji} in a haiku:

\begin{enumarabic}
  \item Add them as custom symbols in the CFG.
    This would work especially well if the positioning
    of the \crim{kigo} and \crim{kireji} is known (e.g. end of sentence $1$
    and end of sentence $3$ as in the examples).
    We can then have the \verb|KIGO| and \verb|KIREJI| symbols
    generate the specific terminals that act as seasonal words
    and cutting words respectively.

  \item Another way could be to have the CFG generate the haiku
    with placeholders for the \crim{kigo} and \crim{kireji},
    and then replace the placeholders with the actual words.
    This would allow a more complicated substitution scheme,
    especially if we want the seasonal and cutting words to be
    particularly relevant to whatever is actually happening in
    the rest of the haiku.
    One could argue that this more flexible approach is no longer
    a CFG, but it could work too.
  
  % algorithm
  \begin{small}
    \begin{answer}
      \centering
      % \begin{minipage}[width=0.6\textwidth]
        \begin{algorithm}[H]
          \caption{Algorithm for generating haikus with kigo and kireji}
          \begin{algorithmic}[1]
            \State \textbf{Input:} CFG for haikus, kigo, kireji
            \State \textbf{Output:} Haiku with kigo and kireji
            \State
            \State kigo\_words $\gets$ \textsc{get\_kigo\_words}() \Comment{All possible/allowed kigo words}
            \State kireji\_words $\gets$ \textsc{get\_kireji\_words}() \Comment{All possible/allowed kireji words}
            \State \Function{generate\_haiku}{CFG, kigo, kireji}
            \State haiku $\gets$ CFG.generate()
            \State kigo $\gets$ \textsc{get\_most\_relevant}(haiku, kigo\_words)
            \State kireji $\gets$ \textsc{get\_most\_relevant}(haiku, kireji\_words)
            \State haiku $\gets$ haiku.update(haiku, kigo, kireji)
            \State \Return haiku
            \EndFunction
          \end{algorithmic}
        \end{algorithm}
      % \end{minipage}
    \end{answer}
  \end{small}
\end{enumarabic}

