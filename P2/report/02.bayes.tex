% define function \sperse that applies \sperse{2}{#1} to its argument
\newcommand{\sperse}[2]{\multirow{#1}{.2\linewidth}{#2}}

\section{Naive Bayes}~\label{sec:introduction}

I implemented Naive Bayes with Laplace smoothing.
Here is the model's performance on the provided dev dataset:

\subsection{Performance on the Positive Class}

\begin{center}
\begin{verbatim}

-------------------------------------------------------------
CLASS: neg
-------------------------------------------------------------

-------------------------------------------------------------
METRICS
-------
         PRECISION    RECALL        F1  ACCURACY  SPECIFICITY
MEASURE   0.747664  0.792079  0.769231      0.76     0.727273
-------------------------------------------------------------

-------------------------------------------------------------
CONFUSION MATRIX
----------------
                 PREDICTED POSITIVE  PREDICTED NEGATIVE
ACTUAL POSITIVE                  80                  27
ACTUAL NEGATIVE                  21                  72
-------------------------------------------------------------
\end{verbatim}
\end{center}

\newpage
\subsection{Performance on the Negative Class}

\begin{center}
\begin{verbatim}

-------------------------------------------------------------
CLASS: pos
-------------------------------------------------------------

-------------------------------------------------------------
METRICS
-------
         PRECISION    RECALL    F1  ACCURACY  SPECIFICITY
MEASURE   0.774194  0.727273  0.75      0.76     0.792079
-------------------------------------------------------------

-------------------------------------------------------------
CONFUSION MATRIX
----------------
                 PREDICTED POSITIVE  PREDICTED NEGATIVE
ACTUAL POSITIVE                  72                  21
ACTUAL NEGATIVE                  27                  80
-------------------------------------------------------------
\end{verbatim}
\end{center}

\subsection{Notes}

\begin{enumarabic}
  \item The ``accuracy'' metric is invariant between different classes.
    Indeed, it's definition does not depend on the class:
    \begin{align}
      \text{accuracy} = \frac{\text{all true positives} + \text{all true negatives}}{\text{total}}
    \end{align}
  \item While I do print the confusion matrix twice,
  as you can notice, the values mirror each other.
  In the context of each class (`pos' or `neg'),
  ``positive'' in the confusion matrix is that class and ``negative'' is the other class.
  
  If you consider the ``pos'' class to be positive,
  then this is the resulting confusion matrix:
  
  \begin{figure}[H]
    \begin{center}
      \begin{tabular}{l | c | c}
                                 & \textbf{Predicted Positive} & \textbf{Predicted Negative} \\
        \midrule
        \textbf{Actual Positive} & $72$ & $21$ \\
        \midrule
        \textbf{Actual Negative} & $27$ & $80$ \\
        \midrule     
      \end{tabular}
    \end{center}
  \end{figure}
\end{enumarabic}


