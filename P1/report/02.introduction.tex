\section{Introduction}~\label{sec:introduction}
We sought to find a \emph{representative} yet \emph{accurate} sample of the public's opinion on AI.
We considered multiple potential sources of data, and their tradeoffs:
\begin{enumroman}
  \item \emph{Research papers} are the most cutting-edge and factually correct, but they
    tend to dive into detailed exposition of novel
    model architectures and ideas, which is disconnected
    from the public's opinion.
  \item \emph{Social media posts} are most accessible to the public, yet they are often too short to offer a nuanced
    opinionand are not fact-checked so they are prone to inaccuracies.
  \item \emph{News articles} can be sensationalized and biased, but they
    are often longer (than social-media posts), fact-checked, and backed by current events
    and innovations. This keeps them (or at least the good ones) in touch with both
    the public's sentiments about AI/technology and in touch with new innovations.
\end{enumroman}

We decided to use news articles as our target data source, since they are a good compromise.
However, we eventually limited our domain to a few specific news sources
that are both reliable and well-known, as explained in~\cref{sec:data-collection}. 
We then considered potential forms of analysis to use as a lens to study the data:
\begin{enumroman}
  \item \emph{Topic modeling} can be used to identify the most common topics
    in the data. We can then focus on these topics and compare them across years
    or in individual years using methods such as procrusties analysis.
    It can also be insightful to see the most prominent topics in conversations
    in given periods, or how specific topics such as \emph{ethics}
    became more or less emphasized after certain events, such as national elections.
  \item \emph{Sentiment analysis} can be used to identify sentiments or tones
    toward AI and how they change.
    This can be insightful in identifying when the public's attitude toward AI became more positive
    or negative.
  \item \emph{Procrustes analysis} can be used to identify shifts in conversation
    in specific time periods, and help highlight periods of particular interest.
\end{enumroman}
