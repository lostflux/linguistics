\section{ASL Finger Spelling with ConvNets}

\subsection{Neural Network Architecture}

% self.layers = nn.Sequential(
            
% #? first convolutional layer
% nn.Conv2d(1, 32, kernel_size=3, stride=1),
% nn.ReLU(),
% nn.MaxPool2d(kernel_size=2, stride=2),

% #? second convolutional layer
% nn.Conv2d(32, 64, kernel_size=3, stride=1),
% nn.ReLU(),
% nn.MaxPool2d(kernel_size=2, stride=2),

% #? third convolutional layer
% nn.Conv2d(64, 64, kernel_size=3, stride=1),
% nn.ReLU(),
% nn.MaxPool2d(kernel_size=2, stride=2),

% #? flatten the output
% nn.Flatten(),

% #? 128-neurons linear layer
% nn.Linear(64, 128),

% #? relu activation
% nn.ReLU(),

% #? output layer with <classes> neurons
% nn.Linear(128, classes)
% )

The neural network has $3$ convolutional layers, each followed by a
ReLU activation function and a max-pooling layer.

\begin{enumarabic}
  \item The first convolutional layer has $32$ output channels and a kernel size of $3$,
    followed by a \verb|ReLU| layer and a max-pooling layer with a kernel size of $2$.
    \begin{enumroman}
      \item \verb|nn.Conv2d(1, 32, kernel_size=3, stride=1)| \\
        IN: $(1 \times 28 \times 28)$ \\
        OUT: $(32 \times 28 \times 28)$
      \item \verb|nn.ReLU()| \\
        IN: $(32 \times 28 \times 28)$ \\
        OUT: $(64 \times 28 \times 28)$
      \item \verb|nn.MaxPool2d(kernel_size=2, stride=2)| \\
        IN: $(64 \times 28 \times 28)$ \\
        OUT: $(64 \times 28 \times 28)$
    \end{enumroman}

\end{enumarabic}
